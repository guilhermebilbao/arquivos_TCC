This is a technical feasibility study, seeking approval of a station broadcasting on frequency modulation in
S�o Pedro de Alc�ntara. The channel used is available in the PBFM -``Plano B�sico de Distribui��o de Canais de Radiodifus�o Sonora em Frequ�ncia Modulada''- 
(Basic Plan of Distribution Channels Broadcasting Frequency Modulation).
During this project serve the requirements within the scope courseware, applicable to channel 218, class C, which would result in the adoption 
and approval of its use by ``Ag�ncia Nacional de Telecomunica��es'' (
National Agency of Telecommunications),the ANATEL.

For a station to be released by ANATEL to
begin their sound transmission in place, it is necessary to prove that the technical specifications of this station
are following the rules presented in ``Resolu��o N�67, de 12 de novembro de 1998'' (Resolution N�67 of November 12, 1998), and its updates.
This resolution requires that, to release the activation broadcaster, it must meet several technical requirements, valuing signal quality
and avoiding interference between channels. Some of these requirements are directly related to the channel and to which class
it is embedded in the PBFM, also organized and managed by ANATEL.

The Resolution provides, among other information that give aid to the designer, a way to for the preparation of technical studies. This
Is the way that was followed during the development of this study, seeking the results using also Recommendation ITU-R P.1546-1, which brings
forecasting models point coverage area, and it's adopted in place of the contours of the field previously recommended by the resolution.

At the end of this study are presented all the results, definitions and technical specifications needed that would prove the technical viability of
 this broadcast before the requeriments imposed by ANATEL.
\vspace{1cm}

KEYWORDS: broadcasting, FM station, Resolution N� 67; radio transmission; ANATEL.

%%%%%%%%%%%%%%%%%%%%%%%%%%%%%%%%%%%%%%%%%%%%%%%%%%%%%
% Modelo para escrever TCCs, disserta��es e teses utilizando LaTeX, ABNTeX e BibTeX
% Autor/E-Mail: Robinson Alves Lemos/contato@robinson.mat.br/robinson.a.l@bol.com.br
% Data: 19/04/2008 
% Colaboradore(s)/E-Mail(s):
% Caso queira colaborar, entre em contato pelo e-mail e informe altera��es que realizou.
%%%%%%%%%%%%%%%%%%%%%%%%%%%%%%%%%%%%%%%%%%%%%%%%%%%%%