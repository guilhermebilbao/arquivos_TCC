This is a technical feasibility study, seeking approval of a station broadcasting on frequency modulation in
S�o Pedro de Alc�ntara. The channel used is available in the Basic Plan of Distribution Channels Broadcasting on Frequency Modulation.
During this project serve the requirements within the scope courseware, applicable to channel 218, class C, which would result in the release of the use by
  National Agency of Telecommunications, ANATEL.

For a station to be released by ANATEL to
beginning their sound transmission in place, it is necessary to prove that the technical specifications of this station
are observing the rules presented in Resolution (No. 67 RESOLUTION OF 12 NOVEMBER 1998) and its updates.
This resolution requires that, to release the activation broadcaster, it must meet several technical requirements, valuing quality
and avoiding signal interference between channels. Some of these requirements are directly related to the channel and which class
it is embedded in the Basic Plan of Distribution Channels Broadcasting on Frequency Modulation, also organized and managed
by ANATEL.

The Resolution provides, among other information that give aid to the designer, a roadmap for the preparation of technical studies. is this
script that was followed during the development of this study, seeking the results using also Recommendation ITU-R P.1546-1, which brings
forecasting models point coverage area, and is adopted in place of the contours of the field previously used.

At the end of this study are being presented all the results, definitions and technical specifications needed that would prove the viability
before issuing this technical requirements imposed by ANATEL.

%%%%%%%%%%%%%%%%%%%%%%%%%%%%%%%%%%%%%%%%%%%%%%%%%%%%%
% Modelo para escrever TCCs, disserta��es e teses utilizando LaTeX, ABNTeX e BibTeX
% Autor/E-Mail: Robinson Alves Lemos/contato@robinson.mat.br/robinson.a.l@bol.com.br
% Data: 19/04/2008 
% Colaboradore(s)/E-Mail(s):
% Caso queira colaborar, entre em contato pelo e-mail e informe altera��es que realizou.
%%%%%%%%%%%%%%%%%%%%%%%%%%%%%%%%%%%%%%%%%%%%%%%%%%%%%