
\textbf{Informações Básicas}

\begin{itemize}

\item \textbf{Nome da entidade requerente:} Não se aplica, projeto com objetivos didáticos.

\item \textbf{Localização da emissora objeto do estudo (cidade, UF):} São Pedro de Alcântara, SC.

\item \textbf{Propósito do estudo:} Projetar emissora de radiofrequência, com canal já definido no plano básico.

\end{itemize}

\textbf{Características técnicas pretendidas}

\begin{itemize}

\item  \textbf{Frequência de operação (MHz):} 91,5.

\item  \textbf{Nº do canal:} 218.

\item  \textbf{Classe:} C.

\item  \textbf{Tipo de sistema irradiante:} Dipolo 1/2 onda, polarização vertical.

\item  \textbf{Coordenadas geográficas de instalação:} 27° 34' 02.72'' S / 48° 48' 33.71'' O.

\end{itemize}

\textbf{Memória Descritiva}

\textbf{Resumo das características da emissora}

\begin{enumerate}

\item \textbf{Frequência de operação (MHz):} 91,5.

\item \textbf{Nº do canal:} 218.

\item \textbf{Potência de operação do transmissor (kW):}0,150kWrms.

\item \textbf{Classe:} C.

\item \textbf{Modo de operação (monofônico, estereofônico, com ou sem canal secundário):} estereofônico.

\end{enumerate}

\textbf{Sistema irradiante}

\begin{enumerate}

\item \textbf{Tipo de antena (onidirecional ou diretiva):} onidirecional.

\item \textbf{Fabricante e modelo da antena:} IDEAL Antenas Profissionais - Dipolo 1/2 Onda para FM.

\item \textbf{Polarização (horizontal, vertical, circular ou elíptica:)} vertical.

\item \textbf{Ganho máximo em relação ao dipolo de meia-onda:} 4,77 dBd (3 elementos).

\item \textbf{Tipo da estrutura de sustentação (auto-suportada ou estaiada):} auto-suportada.

\item \textbf{Altura física total da estrutura de sustentação em relação à sua base (solo):} 55 metros.

\item \textbf{Altura do centro geométrico da antena em relação à base da estrutura de sustentação (solo):} 59,244m.

\item \textbf{Altitude da base da estrutura de sustentação (solo) sobre o nível do mar:} 285m.

\item \textbf{Altura do centro geométrico da antena sobre o nível médio do terreno:} 55,914m.

\end{enumerate}

\textbf{Linha de transmissão de radiofrequência}

\begin{enumerate}

\item \textbf{Fabricante e modelo:} RFS - 1-5/8'' CELLFLEX° Lite-loss Foam-Dieletric Coaxial Cable.

\item \textbf{Impedância característica:} 50 +/- 1.

\item \textbf{Comprimento total:} 65m.

\item \textbf{Atenuação em dB por 100 metros:} 0.68 dB/100m.

\item \textbf{Eficiência:} 0,569.

\end{enumerate}

\textbf{Informações sobre ERPmax e ERPaz}

\begin{enumerate}

\item \textbf{ERP máxima (kW):} 0, 256kW.

\item \textbf{ERP, por radial (kW):} 

Azimute/dBk: 0°/-8,07; 30°/-9,13; 60°/-9,92; 90°/-10,06; 120°/-9,92; 150°/-9,13; 180°/-8,07; 210°/-7,02; 240°/-6,35; 270°/-5,91; 300°/-6,35; 330°/-7,02.


\end{enumerate}

\textbf{Enquadramento na classe}

\begin{enumerate}

\item \textbf{ERP máxima proposta para cada radial:}

Azimute/dBk: 0°/-8,07; 30°/-9,13; 60°/-9,92; 90°/-10,06; 120°/-9,92; 150°/-9,13; 180°/-8,07; 210°/-7,02; 240°/-6,35; 270°/-5,91; 300°/-6,35; 330°/-7,02.

\item \textbf{ERP máxima proposta para cada radial, corrigida para a altura de referência sobre o nível médio do terreno por radial, para a classe da emissora:}

Azimute/dBk: 0°/-8,07; 30°/-9,13; 60°/-9,92; 90°/-10,06; 120°/-9,92; 150°/-9,13; 180°/-8,07; 210°/-7,02; 240°/-6,35; 270°/-5,91; 300°/-6,35; 330°/-7,02.

\item \textbf{Distância ao contorno de 66 dBm para cada radial:}

Azimute/Km: 0°/10,5; 30°/11; 60°/9; 90°/9; 120°/7; 150°/9; 180°/11; 210°/3,2; 240°/3,4; 270°/3,6; 300°/3,4; 330°/3,2.

\item \textbf{Média aritmética das distâncias ao contorno de 66 dBm :} Aproximadamente 7 km.

\end{enumerate}

\textbf{Situação Geral}

\textbf{Distâncias aos contornos das diversas áreas de serviço, segundo cada radial, de acordo com:}

\begin{enumerate}

\item \textbf{Azimute de orientação em relação ao Norte Verdadeiro:} 270°.

\item \textbf{Altura do centro geométrico da antena com relação ao nível médio de cada radial:}

Azimute/HMNT(m): 0°/185,86; 30°/270,78; 60°/175,10; 90°/178,04; 120°/93,78; 150°/147,38; 180°/192,66; 210°/-50,55; 240°/-157,85; 270°/-235,35; 300°/-67,85; 330°/-61,07.

\item \textbf{Intensidade de campo (dBm):} informações disponibilizadas da tabela~\ref{tabelaGeral}

\item \textbf{Distância aos contornos 1, 2 e 3, em cada radial:} informações disponibilizadas da tabela~\ref{tabelaGeral}

\end{enumerate}

\textbf{Nível Médio do Terreno}

\begin{enumerate}

\item \textbf{Azimute de orientação de cada radial, em relação ao Norte Verdadeiro:}

Radial/Azimute: 1/0°; 2/30°; 3/60°; 4/90°; 5/120°; 6/150°; 7/180°; 8/210°; 9/240°; 10/270°; 11/300°; 12/330°.

\item \textbf{Nível médio de cada radial:}

Radial/NMT(m): 1/158,38; 2/73,46; 3/169,14; 4/166,2; 5/250,46; 6/196,86; 7/151,58; 8/394,8; 9/502,1; 10/579,6; 11/412,1; 12/405,32

\item \textbf{Nível médio do terreno:} 288,33m


\end{enumerate}
