Aqui deve-se entrar com a introdu��o.

Estudo e compreens�o das normas mais recentes em rela��o � transmiss�o de r�dio FM, utilizando ferramentas livres oferecidas oela ANATEL

Realizar um estudo de viabilidade t�cnica de um canal de r�dio em frequ�ncia modulada, baseando-se num cen�rio real. Colocar em pr�tica
os conhecimentos obtidos das recomenda��es, aplicando em situa��o real e poss�vel. Com os resultados obtidos, ser� elaborada uma solu��o para
cada eventual problema que surgir.

Escrever a justificativa apresentada no resumo do tcc1	

Nos primeiros cap�tulos s�o estudados os documentos oficiais aprovados referentes aos c�lculos de viabilidade de um canal digital. Em seguida
, no quarto cap�tulo, � apresentada uma proposta de canais de R�dio FM. Depois � apresentado o canal 238 dispon�vel, na localidade de S�o 
Pedro de Alc�ntara, dispon�vel pela ANATEL. O pr�ximo passo ser� apresentar os c�lculos envolvidos na viabilidade do canal. Ao final, s�o 
apresentados as conclus�es tomadas e novas propostas de trabalhos.

%\chapter{Projetando a Emissora Fm em S�o Pedro de Alc�ntara}



%%%%%%%%%%%%%%%%%%%%%%%%%%%%%%%%%%%%%%%%%%%%%%%%%%%%%
% Modelo para escrever TCCs, disserta��es e teses utilizando LaTeX, ABNTeX e BibTeX
% Autor/E-Mail: Robinson Alves Lemos/contato@robinson.mat.br/robinson.a.l@bol.com.br
% Data: 19/04/2008 
% Colaboradore(s)/E-Mail(s):
% Caso queira colaborar, entre em contato pelo e-mail e informe altera��es que realizou.
%%%%%%%%%%%%%%%%%%%%%%%%%%%%%%%%%%%%%%%%%%%%%%%%%%%%%